\begin{filecontents*}{demo.bib}
@inproceedings{doe2023example,
    title = "An example title",
    author = "Doe, Jane and Mustermann, Erika",
    booktitle = "Proceedings of Some Conference",
    year = "2023",
    address = "Earth",
    publisher = "Fancy Publisher",
    url = "https://example.com",
    pages = "7--11",
}
\end{filecontents*}

% Use XeLaTeX!
\documentclass[aspectratio=32]{beamer}
\usepackage{mainlpbeamer}

% In conjunction with XeLaTeX, use this instead of TIPA (or directly copy-paste the IPA characters, no package needed)
\usepackage{xunicode}


\title{This is my talk}
\author{Jeanne Doe}
\institute{Institute}
\venue{Venue}
\date{Date}

\AtBeginDocument{
  \renewcommand\textipa[2][r]{{\fontfamily{cm#1}\tipaencoding #2}}
}


\begin{document}

\begin{frame}[noframenumbering, plain]
   % \tikz [remember picture,overlay] \draw[help lines] (0,0) grid (10,-20);
   \tikz [remember picture,overlay]
    \node at
        ([xshift=-2.1cm, yshift=-1.8cm]current page.east) 
        {\includegraphics[width=0.2\textwidth]{mainlp-logo-500.png}};
    \tikz [remember picture,overlay]
    \node at
        ([xshift=0.2cm, yshift=-1.68cm]current page.east) % aligning the baselines of 'CIS' and 'nlp'
        {\includegraphics[width=0.2\textwidth]{CIS-logo.png}};
   \titlepage
\end{frame}

\subtitledsection{Here's a section}{with a subtitle}

\begin{frame}{How can we do this?}

\textit{Servus,} this is a \textbf{demo} slide. The theme is still a WIP, but the a b c d e overall look won't change much.

Here's a parenthetical reference \citep{doe2023example}.

Here's a text colour for \textem{emphasis}.
\end{frame}

\subtitledsection{Another section}{with another subtitle}

\begin{frame}{Let's try IPA!}
ɛtəɹ 

d\textsyllabic{n} \textsuperscript{2}fj\ae{}\textlengthmark{}\textrtaild{}\textschwa{} \textsuperscript{1}gl\textepsilon{}\textsubarch{\textsci{}} \textsuperscript{1}o\textlengthmark{} \textsuperscript{1}f\textscripta{}l

Normal text for comparison
\end{frame}

\begin{frame}[allowframebreaks]
    \bibliography{demo}
    \bibliographystyle{vbib} % TODO: stylefile specifically for presentations
\end{frame}

\end{document}
