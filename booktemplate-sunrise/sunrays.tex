%%%%%%%%%%%%%%%%%%%%%%%%%%%%%%%%%
% Frame options:
% - A: inner frame with sun
% - B: outer frame, inverted colours
% - C: outer frame, sun in the centre
%%%%%%%%%%%%%%%%%%%%%%%%%%%%%%%%%
% Ray options:
% - 1: (most dashes)
% - 2
% - 3: (fewest dashes)
%%%%%%%%%%%%%%%%%%%%%%%%%%%%%%%%%
% sunrays-inner.pdf: A2
% sunrays-outer.pdf: B2
% sunrays-title.pdf: C2
%%%%%%%%%%%%%%%%%%%%%%%%%%%%%%%%%
\documentclass{article}
\usepackage{tikz}
\usepackage[
    % paperwidth=13.51cm,
    % paperheight=19.27cm,
    paperwidth=148mm,
    paperheight=210mm,
    margin=0cm,
    ]{geometry}
\definecolor{bgcol}{gray}{0.65}
\definecolor{maincol}{gray}{1}
\newlength\linebreadth
\setlength\linebreadth{0.5mm}

\tikzset{
    busydash/.style={dash=on 4mm off 2mm phase 4.35mm},
    calmdash/.style={dash=on 10mm off 2mm phase 10.35mm},
    calmdash2/.style={dash=on 10mm off 2mm phase 6.3mm},
    }
\begin{document}%
\begin{tikzpicture}[remember picture,overlay,shift={(current page.center)}]
    \begin{scope}
    \clip(-6.755,-9.635) rectangle ++(13.51,19.27);

        %% -----
        %% Option B:
        %% invert colours
        %% continued at the end of the file!!
        % \definecolor{maincol}{gray}{0.25}
        % \definecolor{bgcol}{gray}{1}
        %% -----
        
        %% -----
        %% Options B & C:
        %% background colour in outer frame
        \fill[bgcol] (-6.73,-9.61) rectangle ++(13.46,19.22);
        %% -----
    
        %% -----
        %% Option A:
        %% inner frame
        % \clip(-5.625,-8.025) rectangle ++(11.25,16.05);
        % \fill[bgcol] (-5.6,-8.0) rectangle ++(11.2,16.80);
        %% -----

        %% draw rays
        %% Option 1
        % \foreach \a in {5, 15,...,359}
        %   \draw[calmdash2, line width=\linebreadth, maincol] (\a:1.8) -- (\a:12);
        %% Options 1 & 2
        \foreach \a in {0, 10,...,359}
            \draw[busydash, line width=\linebreadth, maincol] (\a:2.5) -- (\a:12);
        %% Options 2 & 3
        \foreach \a in {5, 15,...,359}
          \draw[line width=\linebreadth,maincol] (\a:1.8) -- (\a:12);
        %% Option 3
        % \foreach \a in {0, 10,...,359}
        %     \draw[calmdash, line width=\linebreadth, maincol] (\a:2.5) -- (\a:12);
          
        %% circles and inner rays
        \draw[maincol, line width=1.2mm] (0,0) circle (18.4mm);
        \fill[maincol] (0,0) circle(17.2mm);
        \foreach \a in {0, 10,...,359}
            \draw[line width=\linebreadth, maincol] (\a:1.8) -- (\a:2.47);
        
        %% draw frames
        \draw[line width=\linebreadth,draw=maincol] (-5.6,-8.0) rectangle ++(11.2,16.0);
        \draw[line width=\linebreadth,draw=maincol] (-6.73,-9.61) rectangle ++(13.46,19.22);
        
        %% -----
        %% Option B:
        % outer frame (hide centre)
        % \fill[bgcol] (-5.575,-7.975) rectangle ++(11.15,15.95);
        %% -----
        
        %% -----
        %% Option C:
        %% outer frame, sun in the centre
        \fill[bgcol] (-5.575,-7.975) rectangle ++(11.15,15.95);
        %% Option 1
        % \foreach \a in {5, 15,...,359}
        %   \draw[calmdash2, line width=\linebreadth, maincol] (\a:1.8) -- (\a:3.7);
        %% Options 1, 2
        \foreach \a in {0, 10,...,359}
            \draw[busydash, line width=\linebreadth, maincol] (\a:2.5) -- (\a:3.66);
        %% Options 2, 3
        \foreach \a in {5, 15,...,359}
          \draw[line width=\linebreadth,maincol] (\a:1.8) -- (\a:3.66);
        %% Option 3
        % \foreach \a in {0, 10,...,359}
        %     \draw[calmdash, line width=\linebreadth, maincol] (\a:2.5) -- (\a:3.7);
        %% circles and inner rays
        \draw[maincol, line width=1.2mm] (0,0) circle (18.4mm);
        \fill[maincol] (0,0) circle(17.2mm);
        \foreach \a in {0, 10,...,359}
            \draw[line width=\linebreadth, maincol] (\a:1.8) -- (\a:2.47);
        %% -----
        
    \end{scope}
\end{tikzpicture}
\end{document}
