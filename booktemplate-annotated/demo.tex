\documentclass[11pt,showtrims]{memoir} % remove the showrtims option before creating the print file
\usepackage{booktemplate-annotated}
\usepackage{blindtext,lipsum}

% /!\ the 'pass' option is required if you don't want to absolutely mess up the margins
% \usepackage[showframe,pass]{geometry}

\renewcommand{\booktitle}{The Book Title}
\renewcommand{\author}{A. N. Author}

\begin{document}

\frontmatter
\vspace*{\fill}
\begin{center}
{\large \textpt{\author}}

\phantom{}\\
{\Huge{\textpt{\AlegreyaSansSCLight \booktitle}}}\raisebox{6pt}{\degree}
\makeatletter
\renewcommand*{\@makefnmark}{}
\makeatother
\footnote{\idx{Author}{book title}\blindtext

\sectionrulefn

\noindent\remark{First Commenter}{\idx{First Commenter}{reader engagement}\blindtext}

\idx{Author}{reader engagement}\blindtext
}
\end{center}


\startmainmatter
\chapter{One}

\newday{Day 1}

\lettrine{L}{orem} \firstline{dolor sit amet, consectetur adipiscing elit, sed} do eiusmod tempor incididunt ut labore et dolore magna aliqua. 
\marginalia{\remark{Qwertyui}{\lipsum[1][3]}\\\maniculel\textit{\lipsum[1][4-6]}\idx{Qwertyui}{lorem ipsum}} 
\lipsum[1][3-4]%
\footnote{%
\idx{Author}{introduction}\lipsum[1] Test.

\sectionrulefnshort

\noindent\blindtext
}

\blindtext

\lipsum[5][1-3]
\marginalia{
\linewithmaniculer{Hello} world. \lipsum[2][1-2]
\idx{Author}{lorem ipsum}
}


\blindtext
\marginalia{General margin comment.}

\blindtext[2]%
\footnote{%
\lipsum[1]\idx{Author}{lorem ipsum}

\remark{Second Commenter}{\lipsum[4]}\idx{Second Commenter}{lorem ipsum}

\remark{Third Commenter}{\lipsum[6][1-2]}\idx{Third Commenter}{lorem ipsum}
}

\blindtext

\Blindtext

\commentpage{%
\noindent\lipsum[3]\idx{Author}{some topic}

\sectionrulefn

\noindent\remark{Other Commenter}{\lipsum[2][1-4]}\idx{Other Commenter}{other topic}

\lipsum[3]\idx{Author}{other topic}

\lipsum[4]

\blindtext[3]
}

\startappendix
\index[comments]{titles \seeonly{book title}|gobbleone}
\index[comments]{readers!zzzzz@\also{reader engagement}|gobbleone}

\sffamily
\footnotesize
\printindex[commenter]
\printindex[comments]

\clearpage
\section{Colophon}
\textit{\booktitle} was first published ..............

This book was designed and typeset by \textsc{the typesetter} in February and March 2022.
The text is set in the following typefaces:

---\textit{Engravers' Oldstyle 205} set in 11~pt with 14~pt leading.
This typeface is based on the \textit{Cochin} face designed by Georges Peignot in 1912, based on the early 18th century engravings by Charles-Nicolas Cochin le Jeune, and was adapted by Sol Hess and Matthew Carter.

---\textit{Alegreya Sans} and \textit{Alegreya Sans SC}: these fonts were designed by Juan Pablo del Peral as part of a super family with humanist design. Comments are set 8/12. For larger titles and drop caps, \textit{Light} variants of the fonts were used.

This book was bound by \_\_\_\_\_\_\_\_\_\_\_\_\_\_\_ in \_\_\_\_\_\_\_\_\_.

\end{document}
